\documentclass{beamer}

\usepackage[utf8]{inputenc}
\usepackage[francais]{babel}
\usepackage{amsthm, amssymb, amsmath}
\usepackage{mathpartir}
\usepackage{amsmath}

\usepackage{xcolor}
\usepackage{listings}
\lstset{
  language=[Objective]Caml,
  %% emph={[2]Nil,Cons,FZe,FSu,ze,su,Ze,Su},
  %% emphstyle={[2]\Constructor},
  %% emph={[3]lookup,failwith},
  %% emphstyle={[3]\Function},
  %% emph={[4]vec,fin,nat,list,tree,completeTree},
  %% emphstyle={[4]\Canonical},
  morecomment=[s]{(*}{*)},
  rangeprefix=\(\*\=,
  rangesuffix=\ \*\),
  includerangemarker=false,
  extendedchars=\true,
  inputencoding=utf8,
  showspaces=false,
  showstringspaces=false,
  showtabs=false,
  basicstyle=\ttfamily\small,
  framesep=4mm,
  moredelim=*[s][\itshape]{(*}{*)},
  moredelim=[is][\textcolor{darkgray}]{§}{§},
  escapechar=°,
  keywordstyle=\color[rgb]{0.627451, 0.125490, 0.941176},
  stringstyle=\color[rgb]{0.545098, 0.278431, 0.364706},
  commentstyle=\color[rgb]{0.698039, 0.133333, 0.133333},
  numberstyle=\color[rgb]{0.372549, 0.619608, 0.627451},
  boxpos=t,
  literate= {'a}{{$\alpha$}}1%
    {->}{{${\to}$}}2
    {*}{{${\times}$}}1
    {::}{{${:\::}$}}1
}

\newcommand{\codefrom}[3]
           {\lstinputlisting[linerange={#3}-End]{../#1/#2.ml}}
\newcommand{\fun}[1]{\lstinline!#1!}

\setbeamertemplate{navigation symbols}{} 

\usetheme{Boadilla}

\title{Titre}

\author{Roman Delgado}

\institute[\textsc{Upmc}]{Université Pierre et Marie Curie}

\date{04/05/2016}


\begin{document}

\begin{frame}

\titlepage

\end{frame}

\begin{frame}

\frametitle{Le $\lambda$-calcul}
\framesubtitle{Présentation formelle}

%voici la définition inductive des lambda termes

\begin{block}{Les $\lambda$-termes}
  \begin{itemize}
  \item $x$ \mbox{variable}
  \item $\lambda x.t$ \mbox{abstraction}
  \item $t\:u$
  \end{itemize}
\end{block}

\begin{block}{}
  $$f(x) = y$$
\end{block}

\end{frame}

%%%%%%%%%%%%%%%%%%%%%%%%%%%%%%%%%%%%%%%%%%%%%%%%%%%%%%%%%%%%%%%%

\begin{frame}[fragile]

\frametitle{Autre titre}

%% Attention, pour écrire du code, il faut mettre "fragile" en option du slide
\begin{lstlisting}
 let test = foo
\end{lstlisting}

\begin{block}{}
  Foo.
\end{block}

\end{frame}

\newcommand{\intg}{\ensuremath{\mathsf{int}}}
\newcommand{\bool}{\ensuremath{\mathsf{bool}}}
\newcommand{\Lam}[2]{\ensuremath{\lambda #1. #2}}
\newcommand{\App}[2]{\ensuremath{#1\:#2}}
\newcommand{\Var}[1]{\ensuremath{#1}}
\newcommand{\Fst}[1]{\ensuremath{#1.\pi_0}}
\newcommand{\Snd}[1]{\ensuremath{#1.\pi_1}}
\newcommand{\Pair}[2]{\ensuremath{(#1, #2)}}
\newcommand{\ifte}[4][]{\ensuremath{\mathsf{if}_{#1}\: #2\: \mathsf{then}\: #3\: \mathsf{else}\: #4}}
\newcommand{\true}{\ensuremath{\mathsf{true}}}
\newcommand{\false}{\ensuremath{\mathsf{false}}}
\newcommand{\zero}{\ensuremath{\mathsf{zero}}} 
\newcommand{\succs}{\ensuremath{\mathsf{succ}}}
\newcommand{\iter}{\ensuremath{\mathsf{iter}}}

\begin{frame}
  \[
  \inferrule[]%
    {\visible<2->{
        \inferrule[]
           {\visible<3->{
               \inferrule[]              
                  {\visible<4->{
                     \inferrule[]
                       {\visible<5->{x:\intg\rightarrow \bool \in E}}
                       {E \vdash x:\intg\rightarrow \bool} \qquad
                     \inferrule[]
                       {\visible<6->{y:\intg\in E}}
                       {E \vdash y:\intg}}}
                  {E \vdash \App{x}{y} : \bool}}}
           {\Delta\vdash \Lam{y}{\App{x}{y}}:\intg \rightarrow \bool}}}
    {\Gamma\vdash \Lam{x}{\Lam{y}{\App{x}{y}}}:(\intg\rightarrow \bool) \rightarrow \intg \rightarrow \bool}
  \]

  \begin{align*}
    \Delta &\triangleq \Gamma,x:\intg\rightarrow \bool \\
    E &\triangleq \Gamma,x:\intg\rightarrow \bool,y:\intg \\
  \end{align*}
\end{frame}


\begin{frame}

\begin{center}
\large{(Backup slides)}
\end{center}

\end{frame}

\begin{frame}
  \ldots
\end{frame}

\end{document}
