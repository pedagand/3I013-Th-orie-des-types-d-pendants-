\documentclass{beamer}

\usepackage[utf8]{inputenc}
\usepackage[francais]{babel}
\usepackage{amsthm, amssymb, amsmath}
\usepackage{mathpartir}
\usepackage{amsmath}

\usepackage{xcolor}
\usepackage{listings}
\lstset{
  language=[Objective]Caml,
  %% emph={[2]Nil,Cons,FZe,FSu,ze,su,Ze,Su},
  %% emphstyle={[2]\Constructor},
  %% emph={[3]lookup,failwith},
  %% emphstyle={[3]\Function},
  %% emph={[4]vec,fin,nat,list,tree,completeTree},
  %% emphstyle={[4]\Canonical},
  morecomment=[s]{(*}{*)},
  rangeprefix=\(\*\=,
  rangesuffix=\ \*\),
  includerangemarker=false,
  extendedchars=\true,
  inputencoding=utf8,
  showspaces=false,
  showstringspaces=false,
  showtabs=false,
  basicstyle=\ttfamily\small,
  framesep=4mm,
  moredelim=*[s][\itshape]{(*}{*)},
  moredelim=[is][\textcolor{darkgray}]{§}{§},
  escapechar=°,
  keywordstyle=\color[rgb]{0.627451, 0.125490, 0.941176},
  stringstyle=\color[rgb]{0.545098, 0.278431, 0.364706},
  commentstyle=\color[rgb]{0.698039, 0.133333, 0.133333},
  numberstyle=\color[rgb]{0.372549, 0.619608, 0.627451},
  boxpos=t,
  literate= {'a}{{$\alpha$}}1%
    {->}{{${\to}$}}2
    {*}{{${\times}$}}1
    {::}{{${:\::}$}}1
}

\newcommand{\codefrom}[3]
           {\lstinputlisting[linerange={#3}-End]{../#1/#2.ml}}
           \newcommand{\fun}[1]{\lstinline!#1!}

\newcommand{\intg}{\ensuremath{\mathsf{int}}}
\newcommand{\bool}{\ensuremath{\mathsf{bool}}}
\newcommand{\Lam}[2]{\ensuremath{\lambda #1. #2}}
\newcommand{\App}[2]{\ensuremath{#1\:#2}}
\newcommand{\Var}[1]{\ensuremath{#1}}
\newcommand{\Fst}[1]{\ensuremath{#1.\pi_0}}
\newcommand{\Snd}[1]{\ensuremath{#1.\pi_1}}
\newcommand{\Pair}[2]{\ensuremath{(#1, #2)}}
\newcommand{\ifte}[4][]{\ensuremath{\mathsf{if}_{#1}\: #2\: \mathsf{then}\: #3\: \mathsf{else}\: #4}}
\newcommand{\true}{\ensuremath{\mathsf{true}}}
\newcommand{\false}{\ensuremath{\mathsf{false}}}
\newcommand{\zero}{\ensuremath{\mathsf{zero}}} 
\newcommand{\succs}{\ensuremath{\mathsf{succ}}}
\newcommand{\iter}{\ensuremath{\mathsf{iter}}}
\newcommand{\subst}[3]{#1[#2 := #3]}
\newcommand{\Inv}[1]{\ensuremath{\mathsf{inv}(#1)}}
\newcommand{\Ann}[2]{\ensuremath{(#1\: :\: #2)}}

\setbeamertemplate{navigation symbols}{} 

\usetheme{Boadilla}

\title{Titre}

\author{Roman Delgado}

\institute[\textsc{Upmc}]{Université Pierre et Marie Curie}

\date{04/05/2016}


\begin{document}


\begin{frame}

\titlepage

\end{frame}

%%%%%%%%%%%%%%%%%%%%%%%%%%%%%%%%%%%%%%%%%%%%%%%%%%%%%%%%%%%%%%%%%%%%

\begin{frame}

\frametitle{Le $\lambda$-calcul non typé}
\framesubtitle{Présentation formelle} 

%voici la définition inductive des lambda termes

\begin{block}{Les $\lambda$-termes}
  \begin{itemize}
  \item $x$ \mbox{variable}
  \item $\lambda x.t$ \mbox{abstraction}
  \item $t\:u$
  \end{itemize}
\end{block}

\begin{block}{Exemple}
  $\Lam{x}{(\Lam{y}{(\App{x}{y})})}\equiv(lambda\:x\:(lambda\:y\:(x\:y)))$
\end{block}

\end{frame}
%%%%%%%%%%%%%%%%%%%%%%%%%%%%%%%%%%%%%%%%%%%%%%%%%%%%%%%%%%%%%%%%%%%%
\begin{frame}
\frametitle{Le $\lambda$-calcul non typé}
\framesubtitle{Réduction et évaluation}

\begin{block}{Reduction}
  $\App{\Lam{x}{x}}{t}\leadsto \subst{x}{x}{t} \leadsto \Var{t}$
\end{block}

\begin{block}{évaluation}
  \begin{align*}
    Trace de lexecution de levaluation dans le programme
    avec commande trace dans utop
  \end{align*} 
\end{block}

\end{frame}

%%%%%%%%%%%%%%%%%%%%%%%%%%%%%%%%%%%%%%%%%%%%%%%%%%%%%%%%%%%%%%%%%%%%
\begin{frame}
\frametitle{Le $\lambda$-calcul non typé}
\framesubtitle{Extensions}

\begin{block}{Les booléens}
  \begin{itemize}
  \item $\true$ 
  \item $\false$
  \item $\ifte{c}{t}{u}$
  \end{itemize}
\end{block}

\begin{block}{Les entiers}
  \begin{itemize}
  \item $\zero$ 
  \item $\succs$ 
  \item $\iter\:n\:f\:a$ 
  \end{itemize}
  $(+\:1\:1) \equiv \Lam{x}{\Lam{y}{\iter\: x\:(\Lam{n}{\succs\:n})\:y  }}$
\end{block}  
\end{frame}
%%%%%%%%%%%%%%%%%%%%%%%%%%%%%%%%%%%%%%%%%%%%%%%%%%%%%%%%%%%%%%%%%%%%
\begin{frame}
\frametitle{Le $\lambda$-calcul simplement typé}
\framesubtitle{Motivations} 

\begin{block}{}
  f = (lambda x (lambda y (ifte x (succ y) y)))
  \begin{itemize}
  \item f 3 true $\rightarrow$ exeption
  \item f true 3 $\rightarrow$ 4  
  \end{itemize}
\end{block}

\begin{block}{}
  Il nous faut un moyen de pouvoir vérifier nos termes avant de les éxécuter
\end{block}



\end{frame}
%%%%%%%%%%%%%%%%%%%%%%%%%%%%%%%%%%%%%%%%%%%%%%%%%%%%%%%%%%%%%%%%%%%%

\begin{frame}
  \frametitle{Le $\lambda$-calcul simplement typé}
  \framesubtitle{Motivations} 
  \[\begin{array}{c@{\qquad}c}  
\boxed{\Gamma\vdash T \ni in}
&
\boxed{\Gamma\vdash ex \in T}
\bigskip\\
%%
&
\inferrule[(Var)]
          {x:T \in \Gamma}
          {\Gamma \vdash x \in T }
\medskip\\
\inferrule[(Abs)]
          {T = A \rightarrow B \\
          \Gamma, x:A \vdash B \ni t}
          {\Gamma \vdash T \ni \Lam{x}{t}}
&
\inferrule[(App)]
          {\Gamma \vdash f \in A\rightarrow B \quad \Gamma\vdash A \ni s}
          {\Gamma\vdash \App{f}{s}\in B}
\medskip\\
%%
\inferrule[(Inv)]
          {\Gamma \vdash t \in T' \:\: T=T'}
          {\Gamma\vdash T \ni \Inv{t}}
&
\inferrule[(Ann)]
          {\Gamma\vdash T \ni t}
          {\Gamma\vdash \Ann{t}{T} \in T}
\end{array}\]
\end{frame}
%%%%%%%%%%%%%%%%%%%%%%%%%%%%%%%%%%%%%%%%%%%%%%%%%%%%%%%%%%%%%%%%%%%%
\begin{frame}
  \frametitle{Le $\lambda$-calcul simplement typé}
  \framesubtitle{Exemple de dérivation de typage}
  
  \[
  \inferrule[]%
    {\visible<2->{
        \inferrule[]
           {\visible<3->{
               \inferrule[]              
                  {\visible<4->{
                     \inferrule[]
                       {\visible<5->{x:\bool\in \Delta}}
                       {\Delta \vdash \bool \ni x} \qquad
                     \inferrule[]
                       {\visible<6->{y:\intg\in \Delta}}
                       {\Delta \vdash y\in\intg}
                  \inferrule[]
                            {\visible<6->{
                                \inferrule[]
                                          {y:\intg\in\Delta}
                                          {\intg\ni y}
                            }}
                           {\Delta\vdash \succs\:y\in\intg}}}
                  {\Delta \vdash \intg \ni \ifte{x}{\succs\:y}{y}   }}}
           {\Gamma\vdash \intg \rightarrow \intg \ni \Lam{y}{\ifte{x}{\succs\:y}{y}}}}}
    {\varnothing\vdash \bool \rightarrow \intg \rightarrow \intg \ni \Lam{x}{\Lam{y}{\ifte{x}{\succs\:y}{y}}}}
    \]

    \begin{align*}
    \Gamma &\triangleq x:\bool \\
    \Delta &\triangleq \Gamma,y:\intg \\
  \end{align*}
  
  

\end{frame}
%%%%%%%%%%%%%%%%%%%%%%%%%%%%%%%%%%%%%%%%%%%%%%%%%%%%%%%%%%%%%%%%%%%%
\begin{frame}
  \frametitle{Les types dépendants}
  \framesubtitle{Intuition}
  
  
\end{frame}
%%%%%%%%%%%%%%%%%%%%%%%%%%%%%%%%%%%%%%%%%%%%%%%%%%%%%%%%%%%%%%%%%%%%
\begin{frame}[fragile]

\frametitle{Autre titre}

%% Attention, pour écrire du code, il faut mettre "fragile" en option du slide
\begin{lstlisting}
 let test = foo
\end{lstlisting}

\begin{block}{}
  Foo.
\end{block}

\end{frame}


%%%%%%%%%%%%%%%%%%%%%%%%%%%%%%%%%%%%%%%%%%%%%%%%%%%%%%%%%%%%%%%%%%%%
\begin{frame}
  \[
  \inferrule[]%
    {\visible<2->{
        \inferrule[]
           {\visible<3->{
               \inferrule[]              
                  {\visible<4->{
                     \inferrule[]
                       {\visible<5->{x:\intg\rightarrow \bool \in E}}
                       {E \vdash x:\intg\rightarrow \bool} \qquad
                     \inferrule[]
                       {\visible<6->{y:\intg\in E}}
                       {E \vdash y:\intg}}}
                  {E \vdash \App{x}{y} : \bool}}}
           {\Delta\vdash \Lam{y}{\App{x}{y}}:\intg \rightarrow \bool}}}
    {\Gamma\vdash \Lam{x}{\Lam{y}{\App{x}{y}}}:(\intg\rightarrow \bool) \rightarrow \intg \rightarrow \bool}
  \]

  \begin{align*}
    \Delta &\triangleq \Gamma,x:\intg\rightarrow \bool \\
    E &\triangleq \Gamma,x:\intg\rightarrow \bool,y:\intg \\
  \end{align*}
\end{frame}

%%%%%%%%%%%%%%%%%%%%%%%%%%%%%%%%%%%%%%%%%%%%%%%%%%%%%%%%%%%%%%%%%%%%
\begin{frame}

\begin{center}
\large{(Backup slides)}
\end{center}

\end{frame}

\begin{frame}
  \ldots
\end{frame}

\end{document}
